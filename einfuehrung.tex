\section{Einführung}

\subsection{Polytope}

\begin{defi}
Eine Menge $K\subseteq\bbR^d$ heißt konvex genau dann wenn $\ul{x},\ul{y}\in K \follows 
[\ul{x},\ul{y}]\subseteq K$ 
\end{defi}
Sind $M,L\subseteq\bbR^d$ konvex so ist auch $M\cap L$ konvex.

\begin{figure}
    \psset{xunit=0.7cm,yunit=0.7cm,runit=0.7cm}
    \centering
    \begin{pspicture}(0,0)(6,5)
        \psccurve(1,1.5)(0,4)(3,5)(3,3)(5,2.5)(3.5,1)(1,1.5) 
        \psline{*-*}(2,4.5)(4.5,2.5)
        \rput(2,4.15){x}
        \rput(4.5,2.15){y}
    \end{pspicture}
    \caption{nicht konvexe Menge}
\end{figure}

\begin{defi}[konvexe Hülle]
Sei $M\subseteq\bbR^d$. Die konvexe Hülle von $M$ ist definierit als:
\begin{align*}
    conv(M):=\bigcap\{K\subseteq\bbR^d: K\mbox{ konvex, }M\subseteq K\}
\end{align*}
\end{defi}
Selbstverständlich ist dann $conv(M)$ konvex.

\begin{defi}[konvexe Kombination]
$\ul{x_1},\dots,\ul{x_n}\in\bbR$, $\lambda_i\geq 0$ sowie $\sum_{i=1}^n\lambda_i=1$ dann ist
die \emph{konvexe Kombination} definiert als: 
\begin{align*}
\ul{x_1}\lambda_1+\ul{x_2}\lambda_2+\dots+\ul{x_n}\lambda_n
\end{align*}
bzw. $\sum_{i=1}^n\ul{x_i}\lambda_i$.
\end{defi}

\begin{satz}
Wenn $K\subseteq\bbR^d$ und $K$ konvex, so gilt:
\begin{align*}
\ul{x_1},\dots,\ul{x_n}\in K \follows \ul{x_1}\lambda_1+\dots+\ul{x_n}\lambda_n\in K
\end{align*}
wobei $\lambda\geq 0$ und $\sum\lambda_i=1$.
\end{satz}

\begin{proof}[Beweisidee]
Dieser Satz kann über eine Induktion über $k$ bewiesen werden. Dazu wird in jedem Schritt 
auf folgende Form umgeformt:
\begin{align*}
(1-\lambda_k)\left(\frac{\lambda_1}{1-\lambda_k}\ul{x_1}+\dots+%
                \frac{\lambda_{k-1}}{1-\lambda_k}\ul{x_{k-1}}\right)+\lambda_k\ul{x_k} \in K
\end{align*}
\end{proof}
Damit ist eine mögliche alternative Definition der konvexen Hülle möglich. Aus vorhergehendem Satz
folgt nämlich, dass $\left\{\sum_{i=1}^k\lambda_i\ul{x_i}:\ul{x_i}\in M, \lambda_i\geq0, 
    \sum\lambda_i=1, k\in\bbN\right\}\supseteq conv(M)$. Es ist leicht einzusehen, dass sogar 
Gleichheit gilt. 

\begin{defi}
Ein (konvexes) Polytop ist die konvexe Hülle von endlich vielen Punkten.
\begin{align*}
&\ul{v_1},\dots,\ul{v_n}\in\bbR^d                &P=conv\{\ul{v_1},\dots,\ul{u_n}\} \\
&V=(\ul{v_1},\dots,\ul{v_n})\in\bbR^{d\times n}  & =conv(V) 
\end{align*}
\end{defi}

\begin{bsp}[d-Würfel]
\begin{equation*}
    C_d:=\left\{\ul{x}\in\bbR^d:|x_1|,\dots,|x_d|\leq1\right\}%
    =conv\left\{\begin{pmatrix}1\\1\\\vdots\\1\end{pmatrix},
                \begin{pmatrix}1\\-1\\1\\\vdots\\1\end{pmatrix},
                \dots,
                \begin{pmatrix}-1\\-1\\\vdots\\-1\end{pmatrix} \right\}=conv(\{-1, 1\}^d) 
\end{equation*}
Ein d-Würfel ist als die konvexe Hülle von $2^d$ Punkten wie oben definiert.
\end{bsp}

\begin{bsp}[d-Kreuzpolytop]
In $\bbR^3$ entspricht das 3-Kreuzpolytop genau dem Oktaeder.
\begin{equation*}
C_d^\ast:=conv\left\{\ul{e_1},-\ul{e_1},\ul{e_2},-\ul{e_2},\dots,\ul{e_n},-\ul{e_n}\right\} %
=\left\{\ul{x}\in\bbR^d:|x_1|+\dots+|x_d|\leq1\right\}
\end{equation*}
Wobei $e_i$ den Vektoren der kanonischen Basis von $\bbR^d$ entsprechen.

\begin{figure}
    \centering
    \psset{xunit=1.5cm,yunit=1.5cm,runit=1.5cm}
    \psset{Alpha=60}
    \begin{pspicture}(-1.6,-1.6)(2.2,2.2)
        \pstThreeDCoor[linecolor=black, xMax=2 ,yMax=2, zMax=2, %
                                        xMin=-1.5,yMin=-1.5,zMin=-1.5]
        \pstThreeDLine(1,0,0)(0,0,1)
        \pstThreeDLine(1,0,0)(0,0,-1)
        \pstThreeDLine(1,0,0)(0,1,0)
        \pstThreeDLine(1,0,0)(0,-1,0)
        \pstThreeDLine(-1,0,0)(0,0,1)
        \pstThreeDLine(-1,0,0)(0,1,0)
        \pstThreeDLine(0,0,1)(0,1,0)
        \pstThreeDLine(0,0,1)(0,-1,0)
        \pstThreeDLine(0,0,-1)(0,1,0)
        \pstThreeDLine[linestyle=dashed](-1,0,0)(0,-1,0)
        \pstThreeDLine[linestyle=dashed](-1,0,0)(0,0,-1)
        \pstThreeDLine[linestyle=dashed](0,0,-1)(0,-1,0)
    \end{pspicture}
    \caption{Oktaeder (3-Kreuzpolytop)}
\end{figure}
\end{bsp}

\begin{bsp}[standard d-Simplex]
Der 3-Simplex entspricht in diesem Fall dem Tetraeder.
\begin{equation*}
    \Delta_d:=conv\left\{\ul{e_1},\dots,\ul{e_{d+1}}\right\}=%
    \left\{\ul{x}\in\bbR^{d+1}:x_1+\dots+x_{d+1}=1, x_1,\dots,x_{d+1}\geq0\right\}\subseteq\bbR^{d+1}
\end{equation*}
\begin{figure}
    \centering
    \psset{xunit=1cm,yunit=1cm,runit=1cm}
    \subfigure[1-Simplex]{
        \begin{pspicture}(3,3)    
            \psaxes[ticks=none, labels=none]{->}(0.2,0.2)(2.8,2.8)
            \psline{-}(0.2,2)(2,0.2)
        \end{pspicture}
    }
    \subfigure[2-Simplex]{
        \psset{Alpha=50}
        \begin{pspicture}(-1.5,-1.5)    
        \pstThreeDCoor[linecolor=black, xMax=2 ,yMax=2, zMax=2]
        \pstThreeDTriangle[fillstyle=vlines](1.3,0,0)(0,1.3,0)(0,0,1.3)
        \end{pspicture}
    }
    \caption{Beispiele für d-Simplexe}
\end{figure}
\end{bsp}
